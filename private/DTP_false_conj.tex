\documentclass[11pt,reqno]{amsart}
%%%%%%%%%%%%%%%%%%%%%%%%%%%%%%%%%%%%%%%%%%%%%%%
\usepackage{latexsym,amscd}
\usepackage{amsmath,amssymb,amsthm,url,mathrsfs}
\usepackage{enumerate,stmaryrd,scalefnt}
% \usepackage{tikz}
\usepackage{color}
% \usetikzlibrary{calc}

\usepackage[colorlinks=true,urlcolor=black,linkcolor=black,citecolor=black]{hyperref}

\theoremstyle{plain}
\newtheorem{theorem}{Theorem}[section]
\newtheorem{corollary}[theorem]{Corollary}
%\newtheorem{lemma}[theorem]{Lemma}
\newtheorem{lemma}{Lemma}
\newtheorem{prop}[theorem]{Proposition}
\newtheorem{assumption}[theorem]{Assumption}

\theoremstyle{definition}
\newtheorem{definition}[theorem]{Definition}
\newtheorem{example}[theorem]{Example}
\newtheorem{question}[theorem]{Question}
\newcounter{claim}
\newtheorem{claim}[claim]{Claim}
\newcounter{conjecture}
\newtheorem{conjecture}[conjecture]{Conjecture}
\newtheorem*{Lemma}{Lemma}
\newtheorem*{Prop}{Proposition}

 \theoremstyle{remark}
 \newtheorem*{remark}{Remark}
 \newtheorem*{remarks}{Remarks}
 \newtheorem*{notation}{Notation}

% \numberwithin{theorem}{section}
% \numberwithin{claim}{section}
% \numberwithin{equation}{section}
% \numberwithin{conjecture}{section}

\DeclareMathOperator{\dom}{dom}

\newcommand{\<}{\ensuremath{\langle}}
\renewcommand{\>}{\ensuremath{\rangle}}
\renewcommand{\phi}{\ensuremath{\varphi}}

\newcommand{\lb}{\ensuremath{\llbracket}}
\newcommand{\rb}{\ensuremath{\rrbracket}}

\newcommand{\N}{\ensuremath{\mathbb{N}}}
\newcommand{\bA}{\ensuremath{\mathbf{A}}}
\newcommand{\rel}[1]{\ensuremath{\mathbin{#1}}}
\newcommand{\res}{\ensuremath{\upharpoonright}}  % restriction

% binary operations
\renewcommand{\leq}{\ensuremath{\leqslant}}
\renewcommand{\nleq}{\ensuremath{\nleqslant}}
\renewcommand{\geq}{\ensuremath{\geqslant}}
%\renewcommand{\lneq}{\ensuremath{\lneqslant}}
\renewcommand{\gneq}{\ensuremath{\gneqslant}}
\renewcommand{\ngeq}{\ensuremath{\ngeqslant}}
\newcommand{\ssubnormal}{\ensuremath{\vartriangleleft}}
\newcommand{\subnormal}{\ensuremath{\trianglelefteqslant}}
\newcommand{\supnormal}{\ensuremath{\trianglerighteqslant}}
\newcommand{\notsubnormal}{\ensuremath{\ntrianglelefteqslant}}
\newcommand{\meet}{\ensuremath{\wedge}}
\newcommand{\join}{\ensuremath{\vee}}
\newcommand{\Meet}{\ensuremath{\bigwedge}}
\renewcommand{\Join}{\ensuremath{\bigvee}}

\newcommand{\Eq}{\ensuremath{\operatorname{Eq}}}
\newcommand{\bEq}{\ensuremath{\mathbf{Eq}}}
\newcommand{\Cg}{\ensuremath{\operatorname{Cg}}}
\newcommand{\Sg}{\ensuremath{\operatorname{Sg}}}
\newcommand{\SD}{\ensuremath{\operatorname{SD}}}
\newcommand{\Con}{\ensuremath{\operatorname{Con}}}
\newcommand{\Sub}{\ensuremath{\operatorname{Sub}}}
\newcommand{\Pol}{\ensuremath{\operatorname{Pol}}}
\newcommand{\Clo}{\ensuremath{\operatorname{Clo}}}
\newcommand{\Sym}{\ensuremath{\operatorname{Sym}}}
\newcommand{\core}{\ensuremath{\operatorname{core}}}

\newcommand{\two}{\ensuremath{\mathbf{2}}}
\newcommand{\three}{\ensuremath{\mathbf{3}}}

% \pagestyle{fancy}
% \lhead{{\it A lemma relating intervals in partition lattices}}  \chead{}
%   \rhead{November 12, 2012}

\begin{document}
\title{Dedekind's Transposition Principal for Lattices of Equivalence Relations}
\date{November 15, 2012}
%% First author (Note: The order of the items here is important!)
\author{William DeMeo}
\email{williamdemeo@gmail.com}
\urladdr{http://www.math.sc.edu/~demeow}
\address{Department of Mathematics\\
University of South Carolina\\Columbia 29208\\USA}
%% AMS subject classification; see http://www.ams.org/msc
%% Only one Primary. Possibly several Secondary.
%\subjclass[2010]{Primary: 08A30; Secondary: 08A60, 06B10.} %, 06B15.}

%\thanks{}

%% Keywords and phrases
%\keywords{congruence lattice, finite algebra, finite lattice
%  representations}

\begin{abstract}
We prove a version of Dedekind's Transposition
Principal that holds in  lattices of equivalence relations.
\end{abstract}

\maketitle

In this note we prove a version of Dedekind's Transposition
Principal\footnote{If $L$ is a modular lattice, then for any two elements $a, b\in L$
the intervals $\lb b, a\join b\rb$ and $\lb a\meet b, a\rb$ are isomorphic.
 See \cite{Dedekind:1900}, or~\cite[page 57]{alvi:1987}.} that
holds in all (not necessarily modular) lattices of equivalence 
relations.
Let $X$ be a set and let $\Eq X$ denote the lattice of equivalence relations on $X$.  Given $\alpha,
\beta \in \Eq X$ and $n \in \N$, let 
$\alpha \circ^{n} \beta$ denote the relation $\alpha \circ \beta
\circ \alpha \circ \cdots \circ \beta \circ \alpha$ in which the symbol $\circ$ appears $n-1$ times.  
The relations $\alpha$ and $\beta$ \emph{$n$-permute} provided 
$\alpha \circ^{n} \beta = \beta \circ^{n} \alpha$.
We define the
\emph{interval sublattice of equivalence relations above $\alpha$ and below
  $\beta$} %, denoted $\lb \alpha, \beta \rb$, 
as follows:
\[
\lb \alpha, \beta\rb:= \{\gamma \in \Eq X \mid \alpha \leq \gamma \leq \beta\}.
\]
Let $L$ be a sublattice of $\Eq X$.  Given $\alpha,
\beta \in L$, let $\lb \alpha, \beta \rb_L:= \lb \alpha, \beta \rb \cap L$,
which we call an \emph{interval sublattice of $L$}, or more simply, an \emph{interval of $L$}.
Given 
$\alpha, \beta, \theta \in L$, let  $\lb \alpha, \beta \rb_L^{\theta(n)}$ denote the
set of equivalence relations in the interval $\lb \alpha, \beta \rb_L$  that $n$-permute with
$\theta$.  That is, 
\[
\lb \alpha, \beta \rb_L^{\theta(n)}:= \{\gamma \in L \mid \alpha \leq \gamma \leq
\beta \text{ and } \gamma \circ^{n} \theta = \theta \circ^{n} \gamma\}.
\]
\begin{lemma}
\label{lem:1}
%If $\eta$ and $\theta$ are $n$-permuting relations in $L\leq \Eq X$, then
If $\eta, \theta \in L\leq \Eq X$ and  $\gamma \circ^{n} \theta = \theta
\circ^{n} \gamma$ for some  $n\in \N$, then
% \begin{equation}
%   \label{eq:0}
\[\lb\theta, \eta \join \theta\rb_L \cong \lb\eta \meet \theta, \eta\rb_L^{\theta(n)} \leq 
\lb\eta \meet \theta, \eta\rb_L.
\]
\end{lemma}
%\begin{remarks}
The lemma states that if $\eta$ and $\theta$ $n$-permute, then
the sublattice $\lb\theta, \eta \join \theta\rb_L$ is isomorphic to the lattice,
$\lb\eta \meet \theta, \eta\rb_L^{\theta(n)}$ of relations in $L$ that are below
$\eta$, above $\eta \meet \theta$, and $n$-permute with $\theta$; moreover, $\lb\eta \meet \theta,
\eta\rb_L^{\theta(n)}$ is a sublattice of $\lb\eta \meet \theta, \eta\rb_L$.

To prove this, we need the following generalized version of \emph{Dedekind's 
Rule}:\footnote{In the group theory setting, the well known Dedekind's
  Rule states that if $A, B, C$ are subgroups of a group,
  and $A\leq B$, then we have the following identity of sets: $A(B\cap C) = B
  \cap AC$.}
\begin{lemma}
\label{lem:dedekind}
If $\alpha, \beta, \gamma \in L \leq \Eq X$, and if $\alpha
\leq \beta$, then we have the following identities of subsets of $X^2$:
\begin{equation}
   \label{eq:1}
   \alpha \circ^{n} (\beta \cap \gamma)= \beta \cap (\alpha \circ^{n} \gamma),
 \end{equation}
\begin{equation}
   \label{eq:2}
   (\beta \cap \gamma) \circ^{n} \alpha = \beta \cap (\gamma\circ^{n} \alpha).
 \end{equation}
%%   \label{eq:1}
%%   \alpha \circ (\beta \cap \gamma) = \beta \cap (\alpha \circ \gamma),
%% \end{equation}
%% \begin{equation}
%%   \label{eq:2}
%%   (\beta \cap \gamma) \circ \alpha = \beta \cap (\gamma \circ \alpha).
%% \end{equation}
%% \begin{equation}
%%   \label{eq:3}
%%   \alpha \circ (\beta \cap \gamma) \circ \alpha = \beta \cap (\alpha \circ \gamma \circ \alpha).
%% \end{equation}
%% \begin{equation}
%%   \label{eq:4}
%%   \alpha \circ^{n} (\beta \cap \gamma)= \beta \cap (\alpha \circ^{n} \gamma).
%% \end{equation}
\end{lemma}
\begin{proof}
It is obvious that 
$\alpha \circ^{n} (\beta \cap \gamma) \subseteq \beta \cap (\alpha \circ^n \gamma)$ 
and $ (\beta \cap \gamma) \circ^{n} \alpha \subseteq \beta \cap (\gamma \circ^n \alpha)$. 
We will prove~(\ref{eq:1}) and~(\ref{eq:2}) by establishing the reverse
inclusions simultaneously by induction.
%% Indeed, since $\alpha\leq \beta$, we have
%% \[
%% \alpha \circ (\beta \cap \gamma) \subseteq \alpha \join (\beta \cap \gamma) \leq
%% \beta \join (\beta \cap \gamma) = \beta.
%% \]
In case $n=1$, the reverse inclusions are easy to prove.  For example,
if $(x,y) \in \beta \cap (\alpha\circ \gamma)$, then
since $(x,y) \in \alpha\circ \gamma$, there exists $c\in X$ such that $x\rel{\alpha} c
\rel{\gamma} y$.  To produce $d\in X$ such that 
$x \rel{\alpha} d \rel{(\beta\cap \gamma)} y$, simply note
that $d = c$ works, since %$(c, y) \in  \gamma$ and 
$(x,c) \in \alpha \leq \beta$ implies $c \rel{\beta} x \rel{\beta} y$, so $(c,y) \in
\beta\cap \gamma$. 

For the inductive step, we assume the following identities hold:
\begin{equation}
   \label{eq:3}
   \alpha \circ^{n-2} (\beta \cap \gamma)= \beta \cap (\alpha \circ^{n-2} \gamma),
 \end{equation}
\begin{equation}
   \label{eq:4}
   (\beta \cap \gamma) \circ^{n-2} \alpha = \beta \cap (\gamma\circ^{n-2} \alpha).
 \end{equation}
Fix $(x,y) \in \beta \cap (\alpha\circ^n \gamma)$.
Since $(x,y) \in \alpha\circ^n \gamma$, there exist $c_1, \dots, c_{n-1}\in X$
such that 
$x\rel{\alpha} c_1 \rel{\gamma} c_2 \rel{\alpha} \cdots \rel{\gamma}c_{n-1}
\rel{\alpha} y$.  We must produce $d_1, \dots, d_{n-1}\in X$ such that 
$x \rel{\alpha} d_1 \rel{(\beta\cap \gamma)} d_2 \rel{\alpha} \cdots
\rel{(\beta\cap \gamma)} d_{n-1} 
\rel{\alpha} y$.
In fact, we will see that $d_i = c_i$ works.  Note that $c_1 \rel{\alpha} x
\rel{\beta} y \rel{\alpha} c_{n-1}$.  Thus, $(c_1, c_{n-1}) \in \alpha \circ
\beta \circ \alpha = \beta$, since $\alpha \leq \beta$.  Therefore,  
$(c_1, c_{n-1}) \in \beta \cap (\gamma \circ^{n-2} \alpha)$, and by the inductive
hypothesis we have
$(c_1, c_{n-1}) \in 
  (\beta \cap \gamma) \circ^{n-2} \alpha$.
\vskip2mm
THIS LEMMA MIGHT ACTUALLY BE FALSE.  THE PROOF STILL NEEDS TO BE COMPLETED AND CHECKED.  
%% , since %$(c, y) \in  \gamma$ and 
%% $(x,c) \in \alpha \leq \beta$ implies $c \rel{\beta} x \rel{\beta} y$, so $(c,y) \in
%% \beta\cap \gamma$. 
\end{proof}

\begin{proof}[Proof of Lemma~\ref{lem:1}]
Let $\eta, \theta \in L\leq \Eq X$ be permuting equivalence relations in $L$, 
so $\eta \circ \theta = \theta \circ \eta = \eta \join \theta$.
Consider the mapping
$\phi : \lb\theta, \eta \join \theta\rb_L \rightarrow \lb\eta \meet \theta,\eta\rb$ 
given by 
$\alpha \mapsto \alpha \meet \eta$.  Clearly $\phi$ maps $\lb\theta, \eta \join \theta\rb_L$ into the sublattice
$\lb\eta \meet \theta,\eta\rb_L \leq \lb\eta \meet \theta,\eta\rb$.  Moreover,
it's easy to see that the range of $\phi$
consists of elements of $L$ that permute with $\theta$, so that 
$\phi$ maps 
into $\lb\eta \meet \theta, \eta\rb_L^\theta$.
Indeed, if $\alpha \in \lb \theta, \eta 
\join \theta\rb_L$, then by 
Lemma~\ref{lem:dedekind} we have
$(\alpha \meet \eta) \circ \theta = \alpha \cap (\eta\circ \theta) = \alpha \cap
(\theta \circ \eta) = \theta \circ (\alpha \meet \eta)$.
%Thus, each $\phi(\alpha)$ permutes with $\theta$.

Next, consider the mapping $\psi : \lb \eta\meet \theta, \eta \rb_L^\theta
\rightarrow \lb\theta, \eta\join \theta\rb$ given by $\psi(\alpha) = \alpha
\circ \theta$.  Note that $\psi(\alpha) = \alpha \circ \theta = \alpha \join
\theta$, an element of $L$, since the domain of $\psi$ is a set of
relations in $L$ that permute with $\theta$.
We show that the two maps
\begin{equation}
  \label{eq:phi}
\phi : \lb\theta, \eta \join \theta\rb_L \ni \alpha \longmapsto 
\alpha \meet \eta  \in \lb\eta \meet \theta, \eta\rb_L^\theta
\end{equation}
\begin{equation}
  \label{eq:psi}
\psi :  \lb\eta \meet \theta, \eta\rb^\theta_L \ni \alpha \longmapsto \alpha \circ \theta \in
\lb\theta, \eta \join \theta\rb_L.
\end{equation}
are inverse lattice isomorphisms.  It is clear that these maps are order preserving.
Also, for $\alpha \in \lb\theta, \eta \join \theta\rb_L$ we have, by Lemma~\ref{lem:dedekind},
$\psi\, \phi(\alpha) = (\alpha \meet \eta)\circ \theta = 
\alpha \cap (\eta\circ \theta) = 
\alpha \cap (\eta\join \theta) = \alpha$. For
$\alpha \in \lb\eta \meet \theta, \eta\rb_L^\theta$, we have, by Lemma~\ref{lem:dedekind},
$\phi\, \psi(\alpha) = \phi(\alpha\circ \theta) =  (\alpha
\circ \theta) \meet \eta = \alpha \circ (\theta \meet \eta)$.

To complete the proof of Lemma~\ref{lem:1}, we show that
$\lb\eta \meet \theta, \eta\rb_L^\theta$ is a sublattice of $\lb\eta \meet \theta,
\eta\rb_L$.
Fix $\alpha, \beta \in \lb\eta \meet \theta, \eta\rb_L^\theta$.  We show
\begin{equation}
  \label{eq:i}
  (\alpha \join \beta) \circ \theta \subseteq \theta\circ (\alpha \join \beta),
\end{equation}
and
\begin{equation}
  \label{eq:ii}
  (\alpha \meet \beta) \circ \theta \subseteq \theta\circ (\alpha \meet \beta).
\end{equation}
The reverse inclusions follow by symmetric arguments.

Fix $(x,y) \in (\alpha \join \beta) \circ \theta$.  Then there exist $c \in X$
and $n< \omega$ such that $x \rel{(\alpha \circ^{n}\beta)} c \rel{\theta} y$.
Thus, $(x,y) \in \alpha \circ^{n}\beta \circ \theta$.  Since $\theta$ permutes
with both $\alpha$ and $\beta$, we have $(x,y) \in \theta \circ \alpha
\circ^{n}\beta \subseteq \theta \circ (\alpha \join \beta)$, which proves
(\ref{eq:i}).
Fix $(x,y) \in (\alpha \meet \beta) \circ \theta$.  Then 
$(x,y) \in (\alpha \circ \theta) \cap (\beta \circ \theta) = 
(\theta \circ \alpha ) \cap (\theta \circ \beta)$.  Therefore, there exist $d_1,
\,d_2$ such that 
$x \rel{\theta} d_1 \rel{\alpha} y$ and 
$x \rel{\theta} d_2 \rel{\beta} y$.  Note that $(d_1, y) \in \alpha \leq \eta$ and 
$(d_2, y) \in \beta \leq \eta$, so $(d_1, d_2) \in \eta$.  Also, $d_1
\rel{\theta} x \rel{\theta} d_2$, so $(d_1, d_2) \in \theta$.  Therefore, 
$(d_1, d_2) \in \eta \meet \theta \leq \alpha \meet \beta$.
In particular, 
%$(d_1, d_2) \in \beta$, so $(d_1,y)\in \beta$, so 
$d_1 \rel{\beta} d_2 \rel{\beta} y$, so
$(d_1,y)\in \alpha \meet \beta$.
% \rel{(\alpha \meet \beta)} d_2 \rel{\beta} y$.  It follows that 
% \rel{(\alpha \meet \beta)} d_2 \rel{\beta} y$.  It follows that 
% and we have $d_1 \rel{(\alpha \meet \beta)} d_2 \rel{\beta} y$.  It follows that 
% %$(d_1, y) \in\alpha\meet \beta$, so 
% 
Thus, $x\rel{\theta} d_1 \rel{(\alpha\meet \beta)} y$, which proves~(\ref{eq:ii}).

\end{proof}


\bibliographystyle{spmpsci}
%\bibliographystyle{plain}
\bibliography{wjd}
\end{document}
