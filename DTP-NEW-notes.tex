\documentclass[11pt,reqno]{amsart}
%%%%%%%%%%%%%%%%%%%%%%%%%%%%%%%%%%%%%%%%%%%%%%%
\usepackage{latexsym,amscd}
\usepackage{amsmath,amssymb,amsthm,url,mathrsfs}
\usepackage{enumerate,stmaryrd,scalefnt}
% \usepackage{tikz}
\usepackage{color}
% \usetikzlibrary{calc}

\usepackage[colorlinks=true,urlcolor=black,linkcolor=black,citecolor=black]{hyperref}

\theoremstyle{plain}
\newtheorem{theorem}{Theorem}[section]

\theoremstyle{definition}
\newtheorem{corollary}[theorem]{Corollary}
%\newtheorem{lemma}[theorem]{Lemma}
\newtheorem{lemma}{Lemma}
\newtheorem{prop}[theorem]{Proposition}
\newtheorem{assumption}[theorem]{Assumption}

\newtheorem{definition}[theorem]{Definition}
\newtheorem{example}[theorem]{Example}
\newtheorem{question}[theorem]{Question}
\newcounter{claim}
\newtheorem{claim}[claim]{Claim}
\newcounter{conjecture}
\newtheorem{conjecture}[conjecture]{Conjecture}
\newtheorem*{Lemma}{Lemma}
\newtheorem*{Prop}{Proposition}

 \theoremstyle{remark}
 \newtheorem*{remark}{Remark}
 \newtheorem*{remarks}{Remarks}
 \newtheorem*{notation}{Notation}
\DeclareMathOperator{\dom}{dom}

\newcommand{\<}{\ensuremath{\langle}}
\renewcommand{\>}{\ensuremath{\rangle}}
\renewcommand{\phi}{\ensuremath{\varphi}}

\newcommand{\lb}{\ensuremath{\llbracket}}
\newcommand{\rb}{\ensuremath{\rrbracket}}

\newcommand{\N}{\ensuremath{\mathbb{N}}}
\newcommand{\bA}{\ensuremath{\mathbf{A}}}
\newcommand{\rel}[1]{\ensuremath{\mathbin{#1}}}
\newcommand{\res}{\ensuremath{\upharpoonright}}  % restriction

% binary operations
\renewcommand{\leq}{\ensuremath{\leqslant}}
\renewcommand{\nleq}{\ensuremath{\nleqslant}}
\renewcommand{\geq}{\ensuremath{\geqslant}}
%\renewcommand{\lneq}{\ensuremath{\lneqslant}}
\renewcommand{\gneq}{\ensuremath{\gneqslant}}
\renewcommand{\ngeq}{\ensuremath{\ngeqslant}}
\newcommand{\ssubnormal}{\ensuremath{\vartriangleleft}}
\newcommand{\subnormal}{\ensuremath{\trianglelefteqslant}}
\newcommand{\supnormal}{\ensuremath{\trianglerighteqslant}}
\newcommand{\notsubnormal}{\ensuremath{\ntrianglelefteqslant}}
\newcommand{\meet}{\ensuremath{\wedge}}
\newcommand{\join}{\ensuremath{\vee}}
\newcommand{\Meet}{\ensuremath{\bigwedge}}
\renewcommand{\Join}{\ensuremath{\bigvee}}

\newcommand{\Eq}{\ensuremath{\operatorname{Eq}}}
\newcommand{\bEq}{\ensuremath{\mathbf{Eq}}}
\newcommand{\bEqX}{\ensuremath{\mathbf{Eq(X)}}}
\newcommand{\Cg}{\ensuremath{\operatorname{Cg}}}
\newcommand{\Sg}{\ensuremath{\operatorname{Sg}}}
\newcommand{\SD}{\ensuremath{\operatorname{SD}}}
\newcommand{\Con}{\ensuremath{\operatorname{Con}}}
\newcommand{\Sub}{\ensuremath{\operatorname{Sub}}}
\newcommand{\Pol}{\ensuremath{\operatorname{Pol}}}
\newcommand{\Clo}{\ensuremath{\operatorname{Clo}}}
\newcommand{\Sym}{\ensuremath{\operatorname{Sym}}}
\newcommand{\core}{\ensuremath{\operatorname{core}}}

\newcommand{\two}{\ensuremath{\mathbf{2}}}
\newcommand{\three}{\ensuremath{\mathbf{3}}}
\newcommand{\circone}{\ensuremath{\circ^{1}}}
\newcommand{\circtwo}{\ensuremath{\circ^{2}}}
\newcommand{\circthree}{\ensuremath{\circ^{3}}}
\newcommand{\circfour}{\ensuremath{\circ^{4}}}
\newcommand{\circi}{\ensuremath{\circ^{i}}}
\newcommand{\circj}{\ensuremath{\circ^{j}}}
\newcommand{\circn}{\ensuremath{\circ^{n}}}
\newcommand{\acb}{\ensuremath{\alpha \circn \beta}}
\newcommand{\bca}{\ensuremath{\beta \circn \alpha}}
% \pagestyle{fancy}
% \lhead{{\it A lemma relating intervals in partition lattices}}  \chead{}
%   \rhead{November 12, 2012}

\begin{document}
\title[Permuting relations and permuting subgroups]{Permuting equivalence relations\\
AND\\
permuting subgroups}
\date{May 20, 2013}
%% First author (Note: The order of the items here is important!)
\author{William DeMeo}
\email{williamdemeo@gmail.com}
\urladdr{\url{http://williamdemeo.wordpress.com}}
\address{Department of Mathematics\\
University of South Carolina\\Columbia 29208\\USA}
%% AMS subject classification; see http://www.ams.org/msc
%% Only one Primary. Possibly several Secondary.
%\subjclass[2010]{Primary: 08A30; Secondary: 08A60, 06B10.} %, 06B15.}

% \begin{abstract}
% \end{abstract}

\maketitle
% \setcounter{section}{+1}
% \setcounter{secnumdepth}{2}

\section{Permuting Equivalence Relations}
Let $X$ be a set and let $\bEqX = \<\Eq X, \meet, \join\>$ denote the lattice of equivalence relations on
$X$.  
Given $\alpha, \beta \in \Eq X$, the \emph{meet} of $\alpha$ and $\beta$ is the usual intersection of subsets:
$\alpha \meet \beta = \alpha \cap \beta$. The \emph{join} 
of $\alpha$ and $\beta$,
denoted $\alpha \join \beta$, is the smallest equivalence relation containing
both $\alpha$ and $\beta$.

Given $\alpha, \beta \in \Eq X$, define the \emph{composition} of $\alpha$ and
$\beta$ to be the binary relation
\[
\alpha \circ \beta = \{(x,y)\in X^2 \mid (\exists z\in X) \, x \rel{\alpha} z \rel{\beta} y\}.
\]
Note that
$\circ$ is associative:
$(\alpha \circ \beta) \circ \gamma = \alpha \circ (\beta \circ \gamma)$.

%Let $\alpha \circ^{(1)} \beta = \alpha$ and, for each $n\in \N$, define the
Let $\alpha \circone \beta = \alpha$ and, for each $n\in \N$, define the
$n$-fold composition of $\alpha$ and $\beta$ to be the binary relation
\[
%\alpha \circn \beta = \alpha \circ (\beta  \circ^{n-1} \alpha ).
\alpha \circn \beta = \alpha \circ \beta  \circ^{n-1} \alpha .
\]
For example,
$\alpha \circone \beta = \alpha$; 
$\alpha \circtwo \beta = \alpha\circ \beta$; 
$\alpha \circthree \beta = \alpha\circ \beta \circ \alpha$; etc.
% \begin{align*}
% \alpha \circone \beta &= \alpha\\
% \alpha \circtwo \beta &= \alpha\circ \beta\\
% \alpha \circthree \beta &= \alpha\circ \beta \circ \alpha\\
% %\alpha \circfour \beta &= \alpha\circ \beta \circ \alpha\circ \beta\\
% &\vdots
% \end{align*}

For $\alpha, \beta \in \Eq X$, it is clear that both
$\alpha$ and $\beta$ are contained in $\alpha \circ \beta$, which is contained in
$\alpha \join \beta$.  In general, $\circ$ is not a binary operation on the set
$\Eq X$, since $\alpha \circ \beta$ is not always an equivalence relation; it may
be strictly contained in $\alpha \join \beta$. In fact, we have
\[
\alpha \join \beta = \bigcup_{k\in \N} \alpha \circ^k \beta.
\]

Suppose $X$ is a set and $\alpha, \beta \in \Eq X$.  It is easy
to check that the following facts hold for all positive integers $i$, $j$, and $n$:
\vskip2mm
\noindent 
{\it Fact 1.} If $i < j$, then $\alpha \circi \beta \subseteq \beta \circj \alpha$.
\vskip2mm
\noindent 
{\it Fact 2.} If $\alpha\circn \beta = \beta \circn \alpha$, then 
$\alpha\circ^{n+1} \beta =
\alpha\circn \beta = %\beta \circn \alpha  = 
\beta\circ^{n+1} \alpha$.
\vskip2mm
\noindent 
{\it Fact 3.} If $\alpha\circn \beta = \beta \circn \alpha$, then 
$\alpha\circn \beta = \alpha \join \beta$.
\vskip2mm
\noindent

The relations $\alpha$ and $\beta$ are said to be \emph{permuting}, or
\emph{permutable}, provided $\alpha \circ \beta = \beta \circ \alpha$; they
are \emph{$n$-permuting}, or \emph{$n$-permutable}, if
$\alpha \circ^{n} \beta = \beta \circ^{n} \alpha$.
% Let $X$ be a set and let $\Eq X$ denote the lattice of equivalence relations on $X$.  

The converse of Fact 3 is not true in general.  
(For instance, when $n=3$ it is easy to find
examples where $\alpha \circ \beta \circ \alpha = \alpha
\join \beta$ and yet $\beta \circ \alpha \circ \beta$ is 
strictly contained in $\alpha \circ \beta \circ \alpha$.)
However,
 for all \emph{even} $n>1$ (and trivially for $n=1$) the converse of Fact 3 holds, and we have the
following lemma.

\vskip2mm
\begin{lemma}
\label{lemma:even-permuting-relations}
 If $n$ is a positive even integer, then the following are equivalent:
 \begin{enumerate}[(i)]
 \item $\alpha \circn \beta = \alpha \join \beta$
 \item $\alpha \circn \beta = \beta \circn \alpha$
 \item $\alpha \circn \beta \subseteq \beta \circn \alpha$
 \end{enumerate}
% \[ \alpha\circn \beta = \alpha \join \beta \iff \alpha\circn \beta = \beta \circn \alpha. \]
\end{lemma}
\begin{proof}
  (i) $\Rightarrow$ (ii):  If (i) holds, then $\beta \circn \alpha \subseteq
  \alpha \circn \beta$.  Suppose $(x,y) \in \alpha \circn \beta$.
  (We show $(x,y) \in \beta \circn \alpha$.)  Then, there
  is a sequence $c_1, \dots, c_{n-1}$ such that 
  $x \rel{\alpha} c_1 \rel{\beta} c_2 \rel{\alpha} c_3  \rel{\cdots} %c_{n-2} \rel{\alpha} 
  c_{n-1} \rel{\beta} y$.  
  Using the reversed sequence, $c_{n-1}, \dots, c_1$, we have $(y,x) \in \bca
  \subseteq \alpha \join \beta$.  (Note: this depends on the 
  assumption that $n$ is even.)   Thus $(y,x) \in \alpha \join \beta = \acb$, so
  there is a sequence $d_1, \dots, d_{n-1}$ such that 
  $y \rel{\alpha} d_1 \rel{\beta} d_2 \rel{\alpha} d_3  \cdots %d_{n-2} %\rel{\alpha} 
  d_{n-1} \rel{\beta} x$.
  Again, because $n$ is even, we can use the reversed sequence to arrive
  at $(x,y) \in \bca$, as desired.
\vskip2mm
\noindent
  (ii) $\Rightarrow$ (iii):  obvious.
\vskip2mm
\noindent
  (iii) $\Rightarrow$ (i):  If (iii) holds, then it is not hard to check that
  $\bca$ is an equivalence relation.  Therefore, $\bca = \alpha \join \beta$.
  Now, by symmetry, the argument used above to prove (i) $\Rightarrow$ (ii) can
  be applied to show that (ii) holds, so we have $\acb = \bca = \alpha \join \beta$.

\end{proof}






%% Given $\alpha,
%% \beta \in \Eq X$, we define the
%% \emph{interval sublattice of equivalence relations above $\alpha$ and below
%%   $\beta$}, denoted $\lb \alpha, \beta \rb$, as follows:
%% \[
%% \lb \alpha, \beta\rb:= \{\gamma \in \Eq X \mid \alpha \leq \gamma \leq \beta\}.
%% \]
%% Let $L$ be a sublattice of $\Eq X$.  Given $\alpha,
%% \beta \in L$, let $\lb \alpha, \beta \rb_L:= \lb \alpha, \beta \rb \cap L$,
%% which we call an \emph{interval sublattice of $L$}, or more simply, an \emph{interval of $L$}.
%% Given 
%% $\alpha, \beta, \theta \in L$, let  $\lb \alpha, \beta \rb_L^\theta$ denote the
%% set of equivalence relations in the interval $\lb \alpha, \beta \rb_L$  that permute with
%% $\theta$.  That is, 
%% \[
%% \lb \alpha, \beta \rb_L^\theta:= \{\gamma \in L \mid \alpha \leq \gamma \leq
%% \beta \text{ and } \gamma \circ \theta = \theta \circ \gamma\}.
%% \]


\section{Permuting Subgroups}
Let $G$ be a group, and let $H, K \leq G$; i.e., $H$ and $K$ are subgroups of
$G$.  Define the \emph{composition} of $H$ and $K$ to be the set\footnote{The
  notation $H\circ K$ is non-standard.  This set is usually called $HK$, but the
  $\circ$ notation works well in the context of these notes.}
\begin{equation}
  \label{eq:1}
H\circ K = \{hk \mid h\in H, \, k\in K\},  
\end{equation}
where juxtaposition, $hk$, denotes group multiplication.
Let $H\circ^1 K = H$ and, for each $n\in \N$, define
\begin{equation}
  \label{eq:2}
H\circn K = H\circ K \circ^{n-1} H.  
\end{equation}
Note that the sets defined in~(\ref{eq:1}) and~(\ref{eq:2}) are not necessarily
groups, but it is easy to very that $H\circ K$ is a group if and only if $H
\circ K = K\circ H$. 
% , in which case we say that $H$ and $K$
% are \emph{permuting subgroups}, or $H$ and $K$ \emph{permute}.

The subgroups $H$ and $K$ are said to \emph{permute} (to be \emph{permuting}, or
to be \emph{permutable}) provided $H \circ K= K \circ H$; they
are \emph{$n$-permuting}, or \emph{$n$-permutable}, provided 
$H \circ^{n} K = K \circ^{n} H$.

Recall that if $G$ is a group and if $H$ is a (possibly trivial) subgroup of $G$, then
the transitive G-set
$\bA = \<G/H, \bar{G}\>$ is the algebra with universe $G/H = $ the left cosets
of $H$ in $G$, and basic operations $\bar{G} = \{g^{\bA} : g\in G\}$, where for
each $xH \in G/H$, we have
\[
g^{\bA}(xH) = (gx)H.
\]
In other words, each $g \in G$ acts on the set of left cosets of $H$ by left
multiplication.
A standard result about the algebra $\bA$ is that its congruence lattice, $\Con
\bA$, is isomorphic to an interval in the subgroup lattice of $G$ 
(see~\cite[Lemma 4.12]{alvi:1987}).  More precisely,
\[
\Con \bA \cong \lb H, G\rb :=\{K \mid H \leq K \leq G\}.
\]
The isomorphism $\lb H, G\rb \ni K \mapsto \theta_K \in \Con \bA$ is given by 
\[
\theta_K = \{(xH, yH)\in G/H \times G/H \mid y^{-1}x \in K \}
\]
and the inverse isomorphism $\Con \bA \ni \theta \mapsto K_\theta \in \lb H,
G\rb$ is given by 
\[
K_\theta = \{ g\in G \mid (H, gH) \in \theta \}.
\]
It follows that every lattice property that is true of all congruence lattices
must also be true of all (intervals of) subgroup lattices.  
Moreover, the following is easy to prove:
\begin{lemma}
\label{lemma:permuting-correspondence}
  In $\Con \<G/H, \bar{G}\>$, two congruences $\theta_{K_1}$ and $\theta_{K_2}$
  permute if and only if the corresponding subgroups $K_1$ and $K_2$ permute.
\end{lemma}
Lemma~\ref{lemma:permuting-correspondence} is well known (e.g., it is stated
without proof in~\cite[Lemma 1]{PalfySaxl}), and the following extension is not
much harder to prove: 
\begin{lemma}
\label{lemma:general-permuting-correspondence}
  In $\Con \<G/H, \bar{G}\>$, two congruences $\theta_{K_1}$ and $\theta_{K_2}$
  $n$-permute if and only if the corresponding subgroups $K_1$ and $K_2$ $n$-permute.
\end{lemma}
\begin{proof}
  (coming soon)
\end{proof}
By the correspondence between subgroups and congruences of G-sets described above, and by
Lemma~\ref{lemma:general-permuting-correspondence}, Facts 1--3 above imply
that for all subgroups $H, K$ of a group $G$, and for all positive integers $i$,
$j$, and $n$, we have
\vskip2mm
\noindent 
{\it Fact 1'.} If $i < j$, then $H \circi K \subseteq K \circj H$.
\vskip2mm
\noindent 
{\it Fact 2'.} If $H \circn K = K \circn H$, then 
$H \circ^{n+1} K =
H \circn K = %K \circn H  = 
K \circ^{n+1} H$.
\vskip2mm
\noindent 
{\it Fact 3'.} If $H\circn K = K \circn H$, then 
$H\circn K = H \join K$.
\vskip2mm
\noindent
Moreover, we have the following subgroup version of Lemma 1:
%% \begin{lemma}
%% \label{lemma:even-permuting-relations}
\vskip2mm
\noindent
{\bf Lemma 1'.} If $n$ is a positive even integer, then the following are equivalent:
 \begin{enumerate}[(i)]
 \item $H \circn K = H \join K$
 \item $H \circn K = K \circn H$
 \item $H \circn K \subseteq K \circn H$
 \end{enumerate}
% \[ \alpha\circn \beta = \alpha \join \beta \iff \alpha\circn \beta = \beta \circn \alpha. \]
%\end{lemma}


% \bibliographystyle{spmpsci}
\bibliographystyle{plain}
\bibliography{wjd}
\end{document}
